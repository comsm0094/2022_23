\documentclass[12pt]{article}
\usepackage{amsfonts, epsfig}
\usepackage[authoryear]{natbib}
\usepackage{graphicx}
\usepackage{fancyhdr}
\pagestyle{fancy}
\lfoot{\texttt{comsm0094.github.io}} \lhead{LC\&B - 01.3 Types of neurons and synapses - Conor} \rhead{\thepage} \cfoot{}
\begin{document}

\section*{Types of neurons and synapses}

This is just a short note to emphasis an important fact about the
brain: there are lots of different types of neurons and lots of
different types synapses.

\subsection*{Dale's principle}

We saw that synapses come in two broad classes, excitatory synapses
which make the post-synaptic neuron more likely to fire a spike and
inhibitory, which make it less likely. Perhaps surprisingly, a given
neuron will only ever have one type of synapse; this is known as
\textsl{Dale's principle}. It means that neurons are also divided into
two main types, \textsl{excitatory neurons} whose synapses are all
excitatory and \textsl{inhibitory neurons} whose synapses are all
inhibitory. This division into excitatory and inhibitory neurons
appears to be a substantial difference between neurons in the brain
and the nodes, a sort of idealized neuron, that make up typical neural
networks. It is important to understand that the division relates to
the out-going signals, neurons usually receive a mixture of excitatory
and inhibitory signals.

In the cortex; the part of the brain which most obviously
distinguishes mammals from simpler animals, the inhibitory neurons
tend to be small and to have only local connectivity; they are also
pretty diverse with lots of different types, basket cells, stellate
cells, fast-spiking PV cells. The excitatory neurons are usually
larger, they tend to have local and distal connections and come in
varieties of one main type: the pyramidal cell. It is tempting to
think of the pyramidal cells as doing `the work' and the inhibitory
cells as helping modulate and sculpt the activity of the pyramidal
cells. It remains to be seen if that is a useful way of thinking of
what happens. In other parts of the brain there are circuits where the
\textsl{principle cells}, the ones tasked with signalling outside of
the region, are inhibitory.


\subsection*{More types of synapses}

Synapses are also classified by the sort of neurotransmitter they
produce. For excitatory synapses this is almost always
\textsl{glutamate}, a small amino acid. This does not end the
classification; there are different receptor types for glutamate, the
main ones are called \textsl{GABA} and \textsl{AMPA}; the acronyms are
short for complicated chemical names and we are, here, straying into
more detail than we might need. The main point is that GABA and AMPA
receptors have different behaviours, both in short-term in the time
course of their binding to the transmitter, and in the longer term, in
how they change in number and strength in response to what is
happening at the synapse. A glutamate synapse will usually have a
mixture of ligand-gated channels with GABA and AMPA receptors.

For inhibitory synapses the most common neurotransmitter is called
\textsl{GABA}, again, the acronym is short for a complicated chemical
name. There are also two types of receptor here, but these are classes
of receptors rather than distinct receptors as in the case of
glutamate: the story for inhibitory synapses is more complex even than
the story of excititory synapses: the two classes are
\textsl{ionotropic} receptors and \textsl{metabotropic} receptors with
the metabotropic receptors actually deviating from our description of
how synapses work since they don't directly involve ligand-gated
channels.

Again, we risk straying into the massive complexity of synapses; the
main point of this short subsection is to note that there is a
complicated story and to, perhaps, suggest that the complexity of
synapses and the huge range of behaviours in terms of the temporal
behaviour and the different PSP profiles this produces is an
interesting element of any attempt to compare learning in the brain
and in computers.

\subsection*{Neuromodulation}

We have concentrated so far on electrical signalling; the part of
neuronal computation that involves neurons sending spikes to each
other. There is another system in the brain that is important for
computation: neuromodulation. Neuromodulators are chemicals which can
change the behaviour of a neuron or synapse; there are a lot of
different neuromodulators, but the `big four' are serotonin, dopamine,
acetylcoline and noradreneline. Neuromodulators are released, at
synapses, by specialised cells which are usually found in specific
brain regions but with axons that spread over the whole
brain. Sometimes the neuromodulator is released to one post-synaptic
cell, often they are released into the extracellular fluid so that
they affect a group of cells. There are a large number of different
receptors for the neuromodulators and these can have complicated
affects, changing the excitability of a neuron for example, or
prompting a synapse to change strength.

It is common to think of neuromodulation as adjusting the
computational circuit, like a series of knobs and levers which can up-
or down-regulate the computational dynamics and they do so over
different timescales. These neuromodulators are very interesting
because they seem to link the quite low-level details of how circuits
work, they are produced by neurons in response to signals to those
neurons, and high-level behaviours. Changes in neuromodulation can be
linked to different decision-making strategies and can, it is thought,
be experienced as mood.

\end{document}

